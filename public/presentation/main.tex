\documentclass{beamer}
\usetheme{metropolis}

\title{FETRANS Lab \\— Simulador de Fenômenos de Transporte}
\subtitle{
  \ \ \ \ \ \ \href{https://fetrans-lab.vercel.app}{fetrans-lab.vercel.app}
}
\author{Régis, Luan}
\date{Novembro de 2025}

% Todo:
% - Adicionar screenshots nos slides de demonstração
% - Aprofundar explicação teórica durante demonstração. Com foco em:  Descrição dos conceitos usados (Breve), descrição de como os conceitos são aplicados, e descrição das principais fórmulas e variáveis usadas em cada tela (breve).
% - Adicionar passos futuros.

\begin{document}

% Slide: título
\frame{\titlepage}

% Índice/roteiro
\begin{frame}{Índice}
  \tableofcontents
\end{frame}

\section{Introdução}
\begin{frame}{Introdução e resumo}
  \begin{itemize}
    \item \textbf{Objetivo}: apoiar estudantes na visualização e compreensão de Fenômenos de Transporte.
    \item Interface interativa com simulações, fórmulas e explicações didáticas.
    \item Destaques: tema claro/escuro, componentes React, foco educacional e usabilidade.
  \end{itemize}
\end{frame}

\section{Tecnologias}
\begin{frame}{Tecnologias (breve)}
  \begin{itemize}
    \item \textbf{Front-end}: Next.js 13+, React 18.
    \item \textbf{UI/UX}: Material UI, \texttt{dnd-kit} (drag-and-drop).
    \item \textbf{Visualização}: Chart.js (\texttt{react-chartjs-2}), Recharts.
    \item \textbf{Modelagem numérica}: Hooks (\texttt{useMemo}, \texttt{useState}, \texttt{useEffect}) para simulações em tempo real.
  \end{itemize}
\end{frame}

\section{Assuntos}
\begin{frame}{Assuntos abordados}
  \begin{itemize}
    \item \textbf{Parte 1}: Introdução térmica, trabalho, calor, balanços de massa/energia.
    \item \textbf{Parte 2}: Transferência de calor — condução, convecção, radiação, equivalência elétrica.
    \item \textbf{Parte 3}: Mecânica dos fluidos — pressão hidrostática, forças em superfícies, empuxo, escoamento.
    \item Propriedades dos materiais: $\rho(T),\; c_p(T),\; k(T),\; \mu(T)$.
  \end{itemize}
\end{frame}

\section{Demonstração}
\subsection{Pressão hidrostática}
\begin{frame}{Simulador de Pressão Hidrostática}
  \textbf{Captura de tela do simulador:}
  \includegraphics[width=\textwidth]{../simulador-pressao.png}
  \textbf{Aplicação:} % Objetivo principal do laboratório
  Visualizar a variação da pressão hidrostática com a profundidade em diferentes fluidos.
\end{frame}

\begin{frame}{Simulador de Pressão Hidrostática}
  \textbf{Funcionalidades:}
  \begin{itemize}
    \item Desenha gráfico da pressão em função da profundidade.
    \item Permite comparação entre fluidos (água, óleo, mercúrio).
    \item Conversão de unidades: Pa, kPa, bar, atm, psi.
    \item Interface: seleção de fluido, ajuste de $\rho$, $g$, profundidade e quantidade de pontos no gráfico.
  \end{itemize}
  \vspace{6pt}
  \textbf{Fórmula da pressão hidrostática:}
  \[ P(h) = P_0 + \rho g h \]
  , onde $P_0$ é a pressão na superfície, $\rho$ é a densidade do fluido, $g$ é a aceleração gravitacional e $h$ é a profundidade.
\end{frame}

\subsection{Transferência de calor}
\begin{frame}{Laboratório de Transferência de Calor}
  \textbf{Captura de tela do simulador:}
  \includegraphics[width=\textwidth]{../heat-transfer-lab.png}
  \textbf{Aplicação:} % Objetivo principal do laboratório
  Explorar os três modos de transferência de calor: condução, convecção e radiação.
\end{frame}

\begin{frame}{Laboratório de Transferência de Calor - Condução}
  \textbf{Funcionalidades:}
  \begin{itemize}
    \item Simulação de condução em uma placa 1D em estado estacionário.
    \item Entrada de parâmetros: temperaturas nas faces, condutividade térmica $k$, comprimento $L$ e área $A$.
    \item Cálculo do perfil de temperatura e fluxo de calor $\dot Q$.
  \end{itemize}
  \vspace{6pt}
  \textbf{Fórmulas-chave:} \\
  Perfil de temperatura em placa 1D em regime estacionário:
  \[ T(x) = T_1 + (T_2 - T_1)\frac{x}{L}\]
  Gradiente de temperatura, fluxo de calor por unidade de área, e taxa total de transferência de calor:
  \[\dfrac{dT}{dx} = \dfrac{T_2 - T_1}{L}, \quad q'' = -k\dfrac{dT}{dx}, \quad \dot{Q} = q''A \]
\end{frame}

\begin{frame}{Laboratório de Transferência de Calor - Convecção}
  \textbf{Funcionalidades:}
  \begin{itemize}
    \item Simulação de convecção em uma superfície plana.
    \item Entradas: temperatura da superfície $T_s$, temperatura do fluido $T_\infty$, coeficiente de convecção $h$, área $A$.
    \item Cálculo do fluxo convectivo $q''$ e Taxa de calor $\dot{Q}$.
  \end{itemize}
  \vspace{6pt}
  \textbf{Lei de Resfriamento de Newton:}
  \[ \dot{Q} = h \cdot A \cdot (T_s - T_\infty) \]
  Aqui supõe-se  $h$ uniforme e escoamento externo simples.
\end{frame}

\begin{frame}{Laboratório de Transferência de Calor - Radiação}
  \textbf{Funcionalidades:}
  \begin{itemize}
    \item Simulação de troca de calor por radiação. De superfície cinza para ambiente grande.
    \item Entradas: temperaturas $T_1$, $T_2$, emissividade $\varepsilon$ (constante), área $A$.
    \item Cálculo do fluxo radiativo $q''$ e taxa de calor $\dot{Q}$.
  \end{itemize}
  \vspace{6pt}
  \textbf{Lei de Stefan-Boltzmann} aplicada à radiação entre superfície e ambiente:
  \[q'' = \varepsilon \sigma (T_s^4 - T_{\text{sur}}^4)\]
  \textbf{Taxa total de transferência de calor:}
  \[\dot{Q} = q'' A\]
\end{frame}

\subsection{Propriedades dos materiais}
\begin{frame}{Laboratório de Propriedades dos Materiais}
  \textbf{Captura de tela do simulador:}
  \includegraphics[width=\textwidth]{../material-properties-lab.png}
  \textbf{Aplicação:} % Objetivo principal do laboratório
  Visualizar como as propriedades dos materiais variam com a temperatura.
\end{frame}

\begin{frame}{Laboratório de Propriedades dos Materiais}
  \textbf{Funcionalidades:}
  \begin{itemize}
    \item Simulação interativa de propriedades térmicas e físicas de materiais comuns.
    \item Visualização gráfica das propriedades em função da temperatura.
    \item Comparação multidimensional de propriedades entre materiais.
    \item Seleção de materiais: metais, cerâmicas, polímeros, líquidos, e gases.
    \item Seleção de propriedades:
    \begin{itemize}
      \item Densidade $\rho(T)$
      \item Calor Específico $c_p(T)$
      \item Condutividade Térmica $k(T)$
      \item Viscosidade $\mu(T)$
    \end{itemize}
  \end{itemize}
\end{frame}

\subsection{Calculadora de resistência térmica}

\begin{frame}{Calculadora de Resistência Térmica}
  \textbf{Captura de tela do simulador:}
  \includegraphics[width=\textwidth]{../thermal-system.png}
  \textbf{Aplicação:} % Objetivo principal do laboratório
  Calcular a resistência térmica total de sistemas compostos por múltiplas camadas e mecanismos de transferência de calor.
\end{frame}

\begin{frame}{Calculadora de Resistência Térmica}
  \textbf{Funcionalidades:}
  \begin{itemize}
    \item Construção interativa de sistemas térmicos com múltiplas camadas.
    \item Cálculo automático da resistência térmica total $R_{total}$.
    \item Suporte para condução entre camadas e convecção externa.
    % A Implementar: \item Visualização do fluxo de calor $\dot{Q}$ através do sistema.
  \end{itemize}
  \vspace{6pt}
  \textbf{Resistência térmica por condução e convecção:}
  \[ R_{cond} = \frac{L}{kA}, \quad R_{conv} = \frac{1}{hA} \]
  \textbf{Resistência térmica equivalente em série e paralelo:}
  \[ R_{\text{série}} = \sum R_i, \quad R_{\text{paralelo}} = \left(\sum \frac{1}{R_i}\right)^{-1} \]
\end{frame}

\subsection{Jogo de tanques}
\begin{frame}{Jogo: Controle de Tanques Pressurizados}
  \textbf{Captura de tela do simulador:}
  \includegraphics[width=\textwidth]{../multitank-game.png}
  \textbf{Aplicação:} % Objetivo principal do laboratório
  Aplicar conceitos de mecânica dos fluidos e controle em um jogo interativo.
\end{frame}

\begin{frame}{Jogo: Controle de Tanques Pressurizados}
  \textbf{Funcionalidades:}
  \begin{itemize}
    \item Monitore e controle níveis de líquido em múltiplos tanques.
    \item Dimensões e material da comporta ajustáveis.
    \item Eventos aleatórios para aumentar o desafio.
    \item Condição de falha: Nível seco ou sobrepressão por mais de 5s.
    % Tanque tem volume fixo, 15m2 de base, 5m de altura.
  \end{itemize}
  
\textbf{Força hidrostática na comporta:}
  $$F_h = \dfrac{1}{2}\rho g h_{\text{eff}}^2 w$$
\textbf{Escoamento por orifício:}
  $$Q_{\text{out}} = C_d A_{\text{comporta}}\sqrt{2gh}, \quad C_d = 0{,}62$$
\textbf{Balanço de massa:}
  $$\dfrac{dV}{dt} = Q_{\text{in}} - Q_{\text{out}}$$
\end{frame}

\section{Próximos passos e palavras finais}
\begin{frame}{Próximos passos}
  \begin{itemize}
    \item Adicionar suporte a mais materiais, propriedades e customizações.
    \item Aprimorar laboratório de transferência de calor com mais cenários.
    \item Adicionar visualização do perfil de temperatura na Calculadora de resistência térmica.
    \item Adição de mais simulações e tópicos avançados.
    \item Melhorias na interface e usabilidade.
    \item Feedback de usuários para aprimorar a ferramenta.
  \end{itemize}
\end{frame}

\begin{frame}{Palavras finais}
  Com esse trabalho, esperamos contribuir para o aprendizado de Fenômenos de Transporte, oferecendo uma ferramenta interativa e acessível para estudantes e educadores.\\
  \vspace{4pt}
  Agradecemos a atenção de todos!\\
  \vspace{14pt}
  \textbf{Links úteis:}
  \begin{itemize}
    \item Aplicação hospedada em: \texttt{https://fetrans-lab.vercel.app}
    \item Código-fonte no GitHub: \texttt{https://github.com/RegisBloemer/simulador-pressao}
  \end{itemize}
\end{frame}

\end{document}