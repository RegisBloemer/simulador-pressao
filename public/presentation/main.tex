\documentclass{beamer}
\usetheme{metropolis}

\title{FETRANS Lab — Simulador de Fenômenos de Transporte}
\author{Régis, Luan}
\date{\today}

\begin{document}

% Slide: título
\frame{\titlepage}

% Índice/roteiro
\begin{frame}{Roteiro}
  \tableofcontents
\end{frame}

\section{Introdução}
\begin{frame}{Introdução e resumo}
  \begin{itemize}
    \item \textbf{Objetivo}: apoiar estudantes na visualização e compreensão de Fenômenos de Transporte.
    \item Interface interativa com simulações, fórmulas e explicações didáticas.
    \item Destaques: tema claro/escuro, componentes React, foco educacional e usabilidade.
  \end{itemize}
\end{frame}

\section{Tecnologias}
\begin{frame}{Tecnologias (breve)}
  \begin{itemize}
    \item \textbf{Front-end}: Next.js 13+, React 18.
    \item \textbf{UI/UX}: Material UI, \texttt{dnd-kit} (drag-and-drop).
    \item \textbf{Visualização}: Chart.js (\texttt{react-chartjs-2}), Recharts.
    \item \textbf{Modelagem numérica}: Hooks (\texttt{useMemo}, \texttt{useState}, \texttt{useEffect}) para simulações em tempo real.
  \end{itemize}
\end{frame}

\section{Assuntos}
\begin{frame}{Assuntos abordados}
  \begin{itemize}
    \item \textbf{Parte 1}: Introdução térmica, trabalho, calor, balanços de massa/energia.
    \item \textbf{Parte 2}: Transferência de calor — condução, convecção, radiação, equivalência elétrica.
    \item \textbf{Parte 3}: Mecânica dos fluidos — pressão hidrostática, forças em superfícies, empuxo, escoamento.
    \item Propriedades dos materiais: $\rho(T),\; c_p(T),\; k(T),\; \mu(T)$.
  \end{itemize}
\end{frame}

\section{Demonstração}
\subsection{Pressão hidrostática}
\begin{frame}{Simulador de Pressão Hidrostática}
  \textbf{Fórmula principal:}
  \[ P(h) = P_0 + \rho g h \]
  \vspace{6pt}
  \begin{itemize}
    \item Pressão em função da profundidade; comparação entre fluidos (água, óleo, mercúrio).
    \item Conversão de unidades: Pa, kPa, bar, atm, psi.
    \item Interface: seleção de fluido, ajuste de $\rho$, $g$, profundidade e pontos do gráfico.
  \end{itemize}
\end{frame}

\subsection{Transferência de calor}
\begin{frame}{Laboratório de Transferência de Calor}
  \textbf{Condução (placa 1D, estado estacionário)}:
  \[ T(x) = T_1 + (T_2 - T_1)\frac{x}{L}, \quad q'' = -k\frac{dT}{dx} \]
  \vspace{4pt}
  \textbf{Convecção (Lei de Newton)}:
  \[ q'' = h(T_s - T_\infty) \]
  \vspace{4pt}
  \textbf{Radiação (superfície cinza)}:
  \[ q'' = \varepsilon\sigma( T_s^4 - T_{sur}^4 ) \]
  \vspace{6pt}
  \begin{itemize}
    \item Modo de operação selecionável (Condução / Convecção / Radiação).
    \item Entradas: temperaturas, $k$, $h$, $\varepsilon$, área $A$ — gráficos de perfil e valores de $\dot Q$.
  \end{itemize}
\end{frame}

\subsection{Propriedades dos materiais}
\begin{frame}{Laboratório de Propriedades dos Materiais}
  \begin{itemize}
    \item Propriedades dependentes da temperatura: $\rho(T)$, $c_p(T)$, $k(T)$, $\mu(T)$.
    \item Visualizações: séries temporais / variação com $T$ e gráfico radar para comparações.
    \item Aplicação: entender como propriedades influenciam condução, armazenamento de energia e escoamento.
  \end{itemize}
\end{frame}

\subsection{Calculadora de resistência térmica}
\begin{frame}{Calculadora de Resistência Térmica}
  \textbf{Fórmulas-chave}
  \[ R_{cond} = \frac{L}{k}, \quad R_{conv} = \frac{1}{h} \]
  \[ R_{total}^{(serie)} = \sum_i R_i, \quad \frac{1}{R_{eq}^{(paralelo)}} = \sum_i \frac{1}{R_i} \]
  \vspace{6pt}
  \begin{itemize}
    \item Montagem interativa: camadas em série, grupos em paralelo, resistências de contato e convecção.
    \item Resultado exibido para área de 1 m$^2$ com detalhamento por elemento.
  \end{itemize}
\end{frame}

\subsection{Jogo de tanques}
\begin{frame}{Jogo: Controle de Tanques Pressurizados}
  \textbf{Conceitos principais}
  \[ F_h = \tfrac{1}{2}\rho g h_{eff}^2 w, \qquad Q_{out} = C_d A_{orif}\sqrt{2 g h} \]
  \vspace{6pt}
  \begin{itemize}
    \item Mecânica do jogo: controlar comportas (ON/OFF) para evitar sobrepressão e nível seco.
    \item Modelagem: balanço de massa $dV/dt = Q_{in} - Q_{out}$, eventos aleatórios e condições de falha.
    \item Objetivo: manter 10 tanques estáveis por 90 s.
  \end{itemize}
\end{frame}

\section{Conclusão}
\begin{frame}{Conclusão e links}
  \begin{itemize}
    \item Ferramenta educacional para apoiar o ensino de Fenômenos de Transporte.
    \item Código e demo hospedados em: \texttt{https://fetrans-lab.vercel.app}
    \item Próximos passos: adicionar screenshots nas lâminas, compilar PDF e ajustar tempo de apresentação.
  \end{itemize}
\end{frame}

\end{document}